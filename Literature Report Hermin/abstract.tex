%
% Abstract (does not appear in the Table of Contents)
\chapter*{Abstract}%

In an aircraft-satellite communications, the aircrafts transmit highly focused beams that are steered exactly towards the satellites. Conventional solution for this problem is to steer the dish antennas mechanically in order to direct the beam. Optical Beamforming Networks (OBFNs) instead, rely on many small and flat antennas that are coordinated in order to generate highly focused beams. In this project, a new type of OBFN proposed by \citet{Meijerink1} will be considered. The correct tuning of these OBFNs such that the beam is steered correctly is a difficult nonlinear problem that, so far, has only been addressed with off-the-shelf solvers in very small setups.

The problem of tuning a large-scale OBFN is very similar to training a neural network. For the later problem, many recent advances have been made under the umbrella term deep learning. The objective of this graduation project is to explore if and how advances in deep learning can be exploited to train the OBFNs. This can be achieved either by formulating the tuning problem as a learning problem, or by modifying existing algorithms in the area of deep learning. The tuning methodology developed in this project is based on feedback that can be measured in real systems (e.g., output power).

The literature survey is divided into two distinct parts. In the first part, we analyze the current setup and algorithm of OBFN proposed by \citet{Blokpoel} and \citet{Meijerink1} with special emphasis on the power optimality criterion. The analysis covers a whole concept of Antenna System, including Antenna Elements (AEs), OBFN based on binary tree topology, Optical Ring Resonators (ORRs), as well as the ORR's parameters and their influence to the group delay of the signal. The second part uses the theoretical information from the first part to support the implementation of deep learning neural network for solving a more general setup of OBFN without any limitation to small setups. This part handles the implementation of the concept of deep learning neural network to train the tuning of the OBFN.

It is expected that ....
(advantages of using neural network concept rather than current concept)

